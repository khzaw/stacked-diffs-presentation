% Created 2024-11-22 Fri 09:22
% Intended LaTeX compiler: pdflatex
\documentclass[11pt]{article}
\usepackage[utf8]{inputenc}
\usepackage[T1]{fontenc}
\usepackage{graphicx}
\usepackage{longtable}
\usepackage{wrapfig}
\usepackage{rotating}
\usepackage[normalem]{ulem}
\usepackage{amsmath}
\usepackage{amssymb}
\usepackage{capt-of}
\usepackage{hyperref}
\makeatletter
\newcommand{\citeprocitem}[2]{\hyper@linkstart{cite}{citeproc_bib_item_#1}#2\hyper@linkend}
\makeatother
\author{Kaung Htet}
\date{\today}
\title{How to Manage Stacked Diffs with Git}
\hypersetup{
 pdfauthor={Kaung Htet},
 pdftitle={How to Manage Stacked Diffs with Git},
 pdfkeywords={},
 pdfsubject={},
 pdfcreator={Emacs 31.0.50 (Org mode 9.7.11)}, 
 pdflang={English}}
\begin{document}

\maketitle
\tableofcontents

\section{Stacked Diffs}
\label{sec:org6984f5a}
\begin{itemize}
\item Today I am going to talk about stacked diffs.
\item You might have also heard it as stacked diffs/stacked branches/stacked PRs workflows.
\item This talk is more focused on \texttt{How to manage staked diffs/branches?} rather than \texttt{Why?}
\begin{itemize}
\item Most resources online don't really talk about \texttt{how} and point to \texttt{Phabricator}, \texttt{Graphite}., etc
\end{itemize}
\item Talk is also tailored more to our repositories.
\end{itemize}
\section{My Background}
\label{sec:orgeb85d27}
\begin{itemize}
\item Started out with \texttt{svn} (very brief)
\item Mercurial \texttt{(hg)},
\item Git \texttt{(git)}
\end{itemize}
\section{What are Stacked Diffs?}
\label{sec:org5a9d651}
\begin{itemize}
\item A workflow concept that invovles stacking \texttt{a series of small, atomic and dependent changes} on top of one another.
\item Allows code reviews to happen on each small change, making them more efficient and managable.
\item Every \textbf{"commit" (diff)} becomes a code review.
\begin{itemize}
\item That's the Phabricator approach.
\item Trying to do that without tooling in our context is unintuitive and impractical.
\item The essense to have work on top of local \texttt{master} and the tool will create a new remote branch for each commit and a respective new code review ("PR" in our speak) for each commit against \texttt{master}.
\end{itemize}
\begin{verbatim}
master (local):    A---B---C---D---E---F
                           \
                       (origin/master)

- D, E, F becomes individual branches
- D is PR against C
- E is PR against C
- F is PR against C

In your only using Github UI,
- in E PR view, you see everything from D+E.
- in F PR view, you see everything from D+E+F.

Landing PRs.
The merge happens for (C)
Tooling sync back to your local branch and rebase against origin-master
master (local):    A---B---C---D---E---F
                                \
                             (origin/master)

\end{verbatim}
\end{itemize}
\section{Branches in \texttt{Git}}
\label{sec:org247ce01}
\begin{itemize}
\item There is no such thing as "branches"
\item Branches are named references to commits.
\begin{itemize}
\item Essentially 41-byte referecne pointing to a commit hash.
\item Exteremely cheap operation compared to other VCS.
\end{itemize}
\end{itemize}
\subsubsection{Compared to other D-VCS?}
\label{sec:org534f7c4}
\begin{itemize}
\item Git branches are still very flexible since it treats them \textbf{purely as movable pointers to commits}.
\item e.g., \textbf{Mercurial} branches are permanent markers in commit history.
\end{itemize}

\begin{center}
\includegraphics[width=.9\linewidth]{./img/pablo-refs.png}
\end{center}
\section{Stacking branches prone to be extremely painful}
\label{sec:orgf7eb519}
\begin{itemize}
\item In theory, stacked diffs seem like a pleasant workflow.
\item Stacked branches are dependent on each other.
\item Anytime there is an upstream change,
\begin{itemize}
\item Every branch in the rest of the stack needs to be recursively rebased on top of one another to stay in sync again.
\item Child branches from D+2 onwards can be stuck in limbo with if D+1 is merged with \texttt{git merge -{}-{}squash} to \texttt{trunk}.
\end{itemize}
\item Potential of cascading merge conflicts.
\item Can get real messy
\begin{itemize}
\item If you are not crystal clear on what's happening. (With/without tooling).
\end{itemize}
\item This talk will suggest workflows/tooling.
\begin{itemize}
\item Without being clear on what's happening underneath, you are prone to a lot of nested mess.
\end{itemize}
\end{itemize}
\section{True \texttt{merge} vs \texttt{squash merge} vs \texttt{rebase merge}}
\label{sec:org5776794}
\begin{itemize}
\item A true \texttt{merge} brings in no changes but connects the two or more histories with a parent pointer to each.
\begin{itemize}
\item \textbf{Preserves} the complete branch history and shows the relationship between branches.
\end{itemize}
\item \texttt{Squash merge} \textbf{combines} all feature commits into a single commit on the base branch, loses branch relationships.
\item \texttt{Rebase merge} replays commits for linear history, no merge commit, new hashes.
\end{itemize}
\begin{verbatim}
Starting state:
main        A---B---C
                \
feature         D---E---F


Regular Merge:   A---B---C---------M
                     \           /
                      D---E---F

Squash Merge:    A---B---C---S
                     \
                      D---E---F

Rebase Merge:    A---B---C---D'---E'---F'
\end{verbatim}
\section{Problem with \texttt{squash merge}}
\label{sec:orge85476f}
\begin{itemize}
\item \texttt{squash} merges are not true merges.
\item It's simply one independent commit added to the target branch containing all the changes \emph{copied} from the resulting diff of the source + target branches.
\item Without any reference to the source branch, literally except for comments in the commit message.
\item Really against squash merges to trunk.
\item Personally a little iffy about this choice, but it makes sense for us given how we use PRs and branches.
\begin{itemize}
\item At the end of the day the goal of \texttt{master} to have atomic changset in each commit that lands on it.
\end{itemize}
\end{itemize}
\section{Rebase}
\label{sec:org7b5fd74}
\begin{itemize}
\item \texttt{Rebasing} is a core part of maintaing stacked branches.
\item \textbf{3 Principles} to follow generally.
\begin{itemize}
\item Keep branches small
\item \texttt{Rebase} on top of \uline{trunk} to keep it up to date.
\item You want each branch to always be on top of each other \texttt{in a linear manner}.
\end{itemize}
\end{itemize}
\subsubsection{Start a \texttt{feature} branch from \texttt{master}}
\label{sec:org4422e5a}
\begin{center}
\includegraphics[width=.9\linewidth]{img/rebase-feature-branch-from-master.png}
\label{org801a71e}
\end{center}
\subsubsection{\texttt{master} has updates}
\label{sec:org04c3cdc}
\begin{center}
\includegraphics[width=.9\linewidth]{img/rebase-feature-branch-master-has-updates.png}
\label{org03a6a70}
\end{center}
\subsubsection{\texttt{rebase} feature onto \texttt{master}}
\label{sec:orgd0bf04c}
\begin{verbatim}
git fetch origin master:master # pull down changes
git rebase master
\end{verbatim}

\begin{center}
\includegraphics[width=.9\linewidth]{img/rebase-feature-onto-master.png}
\label{org7d48271}
\end{center}
\section{Making stacked branches}
\label{sec:org2f57e96}
\subsubsection{Create stacked branches}
\label{sec:org2e6c077}
\begin{center}
\includegraphics[width=.9\linewidth]{img/create-stacked-branches.png}
\label{orgc2226b4}
\end{center}
\subsubsection{Keep your changset small}
\label{sec:org1bd7772}
\begin{itemize}
\item Say you keep working on \texttt{feature1}
\end{itemize}
\begin{verbatim}
git checkout feature1
# ... more work here
git commit -am 'Debug'
# ... more work here
git commit -am 'Fix typo'
git revert HEAD^
\end{verbatim}

\begin{center}
\includegraphics[width=.9\linewidth]{img/feature-1-has-updates.png}
\label{org25d07ff}
\end{center}

\begin{itemize}
\item \texttt{rebase} feature1
\end{itemize}
\begin{verbatim}
git rebase -i # DEMO?
\end{verbatim}
\begin{center}
\includegraphics[width=.9\linewidth]{img/rebase-feature-1.png}
\label{org6221114}
\end{center}

\begin{itemize}
\item In fact keep it as lean and atomic as possible
\begin{itemize}
\item Not necessarily mean literally one commit
\end{itemize}
\end{itemize}

\begin{center}
\includegraphics[width=.9\linewidth]{img/rebase-feature-1-clean.png}
\label{org52dde7f}
\end{center}
\section{\texttt{Rebase} child branches one by one}
\label{sec:org9f3a1f5}
\begin{verbatim}
git checkout feature2
git rebase feature1
\end{verbatim}
\begin{center}
\includegraphics[width=.9\linewidth]{img/rebase-feature2-onto-feature1.png}
\label{org7fafbcd}
\end{center}

\begin{verbatim}
git checkout feature3
git rebase feature2
\end{verbatim}
\begin{center}
\includegraphics[width=.9\linewidth]{img/rebase-child-branches-final-state.png}
\label{orge572f6b}
\end{center}
\section{Can we do it in one go?}
\label{sec:org4990027}
\begin{itemize}
\item Yes, enter \texttt{git rebase -{}-{}onto} for surgically more precise rebasing.
\end{itemize}

\begin{verbatim}
git rebase --onto <newparent> <old parent> HEAD
git rebase --onto <newparent> <old parent> <until>
\end{verbatim}
\subsubsection{Rewind the state after \texttt{rebasing} feature1}
\label{sec:orga545ba4}
\begin{center}
\includegraphics[width=.9\linewidth]{img/rebase-feature-1-clean-with-head.png}
\label{orgfb09df7}
\end{center}
\subsubsection{Go to the innermost child branch at the end of the stack.}
\label{sec:orgd2fe6a2}
\begin{verbatim}
git checkout feature4
git rebase --onto feature1 feature2^ feature4 --update-refs
\end{verbatim}
\begin{center}
\includegraphics[width=.9\linewidth]{img/stacked-branches-final-state.png}
\label{org48649c2}
\end{center}
\section{Merging \texttt{feature1} to \texttt{master}.}
\label{sec:orga2a1415}
\subsubsection{feature1 has been reviewed and finally merged to \texttt{master}.}
\label{sec:org5b18b8f}
\begin{verbatim}
git checkout master
git merge --squash feature1
# At github remote, it auto delete the remote branch (origin/feature1) -- Our settings
\end{verbatim}

\begin{center}
\includegraphics[width=.9\linewidth]{img/feature-1-post-squash-merge-to-master.png}
\label{orgd916233}
\end{center}
\subsubsection{Now what?}
\label{sec:orgda474e6}
\begin{itemize}
\item Your local feature1 branch still exits.
\item After pulling, Git does not even know that it was merged to master
\item When you delete, you have to use \texttt{-D} instead of \texttt{-d}.
\item Anyway, let's continue the rebase.
\end{itemize}
\begin{verbatim}
git branch -D feature1
git checkout feature3
git rebase --onto origin/master feature2^ feature3 --update-refs

# still need to update the remote refs
git push --force-with-lease origin feature3:feature3
git push --force-with-lease origin feature2:feature2
\end{verbatim}

\begin{center}
\includegraphics[width=.9\linewidth]{stack-branches-final-state-update-refs.png}
\label{org95d0a76}
\end{center}
\section{What exactly is \texttt{-{}-{}update-refs} ?}
\label{sec:orgabe3f09}

\subsubsection{Go back to post squash merge state}
\label{sec:org28a6f85}
\begin{center}
\includegraphics[width=.9\linewidth]{post-squash-merge-feature-1.png}
\label{orgf5ca200}
\end{center}
\subsubsection{Without \texttt{-{}-{}update-refs}}
\label{sec:org3ab16ce}
\begin{itemize}
\item What you wanted still happens, old branch references still remain
\end{itemize}
\begin{verbatim}
git checkout feature3
git rebase --onto origin/master feature2^ feature3 --udpate-refs
git reset --hard origin/feature3

\end{verbatim}

\begin{center}
\includegraphics[width=.9\linewidth]{rebase-child-branches-without-stacked-refs.png}
\label{orgc2b4497}
\end{center}
\section{To sum up}
\label{sec:org7e16779}

\begin{itemize}
\item Keep changset small in a branch. Always clean it up. (\texttt{git rebase -i})
\item If a branch is merged, use \texttt{git rebase -{}-{}onto} with \texttt{-{}-{}update-refs} from the innermost branch
\end{itemize}

\begin{verbatim}

# after every merge, go to innermost branch
git rebase --onto <newparent> <oldparent> <until> --update-refs


# update the remote refs
for all branches from branch+1 to innermost branch
  git push -f origin <ref>
endfor
\end{verbatim}
\subsubsection{Can I automate this part.}
\label{sec:orgc9e9501}
\begin{itemize}
\item Yes, check out \href{https://www.git-town.com/}{Git Town}.
\end{itemize}
\begin{verbatim}
git town hack <my-feature>
# work on stuffs and create commits
git town append <branch2>
git town append <branch3>
git town append <branch4>
git town append <branch5>

# Ship branch2 for example
git town ship branch2

# And then from your youngest branch
git town sync
\end{verbatim}
\end{document}
